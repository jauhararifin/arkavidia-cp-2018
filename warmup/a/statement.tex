
\documentclass{article}
\usepackage{amsmath}
\usepackage{listings}
\usepackage[margin=1in]{geometry}
\pagenumbering{gobble}
\setlength{\tabcolsep}{10pt}
\renewcommand{\arraystretch}{1.7}
\usepackage{indentfirst}
 
\begin{document}

\section*{\hfil A. Penjumlahan\hfil}

\begin{center}
\begin{tabular}{ |cc| } 
 \hline
 Time Limit & 1 detik \\ 
 \hline
 Memory Limit & 256 MB \\
 \hline
\end{tabular}
\end{center}

\subsection*{Deskripsi}

\par\noindent Diberikan dua buah bilangan bulat $A$ dan $B$, tentukan nilai $A + B$.

\subsection*{Format Masukan}

\par\noindent Baris pertama berisi sebuah bilangan bulat $T$ yaitu banyaknya kasus.
\par\noindent $T$ baris berikutnya berisi dua buah bilangan yaitu $A$ dan $B$.

\subsection*{Format Keluaran}

\par\noindent Setiap kasus keluarkan sebuah bilangan yaitu nilai $A + B$.

\subsection*{Contoh Masukan}

\begin{lstlisting}
2
1 2
3 4
\end{lstlisting}

\subsection*{Contoh Keluaran}

\begin{lstlisting}
3
7
\end{lstlisting}


\subsection*{Batasan}

\begin{itemize}
	\item $1 \leq T \leq 100$
	\item $-10^9 \leq A, B \leq 10^9$
\end{itemize}

\end{document}
