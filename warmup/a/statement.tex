
\documentclass{article}
\usepackage{amsmath}
\usepackage{listings}
\usepackage[margin=1in]{geometry}
\pagenumbering{gobble}
\setlength{\tabcolsep}{10pt}
\renewcommand{\arraystretch}{1.7}
\usepackage{indentfirst}
 
\begin{document}

\section*{\hfil A. Kakek Sugih\hfil}

\begin{center}
\begin{tabular}{ |cc| } 
 \hline
 Time Limit & 1 detik \\ 
 \hline
 Memory Limit & 256 MB \\
 \hline
\end{tabular}
\end{center}

\subsection*{Deskripsi}

\par\indent Kakek Sugih mempunyai $N$ permen yang berbeda. Ia mempunyai 2 cucu yang bernama Rutfa dan Harma. Kakek Sugih adalah orang yang sangat dermawan sehingga beliau sering menyisihkan sebagian permennya untuk diberikan ke panti asuhan. 

\par Kakek Sugih akan menyisihkan $X$ permen untuk diberikan ke panti asuhan, $Y$ permen untuk Rutfa, dan $(N - X - Y)$ permen untuk Harma. Kakek Sugih tidak mungkin memberikan permen sejumlah bilangan negatif. Kakek Sugih tidak ingin kedua cucunya bertengkar, sehingga Kakek Sugih harus dapat membagikan permen berjumlah sama kepada masing-masing Rutfa dan Harma.

\par Kakek Sugih penasaran, berapakah banyak cara pembagian permen yang mungkin? Pembagian permen dikatakan berbeda apabila ada setidaknya satu permen yang diberikan kepada subjek yang berbeda. 

\par Karena banyaknya cara pembagian permen bisa sangat besar, ia meminta Anda untuk memberitahu hasilnya modulo $1.000.000.007$.

\subsection*{Format Masukan}
\par\noindent Baris pertama berisi sebuah bilangan bulat $N$.

\subsection*{Format Keluaran}

\par\noindent Sebuah baris berisi banyaknya cara pemilihan permen yang berbeda.

\subsection*{Contoh Masukan}

\begin{lstlisting}
4
\end{lstlisting}

\subsection*{Contoh Keluaran}

\begin{lstlisting}
19
\end{lstlisting}


\subsection*{Batasan}

\begin{itemize}
	\item $0 \leq N \leq 10$
\end{itemize}

\end{document}
