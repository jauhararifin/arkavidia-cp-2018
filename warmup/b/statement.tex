
\documentclass{article}
\usepackage{amsmath}
\usepackage{listings}
\usepackage[margin=1in]{geometry}
\pagenumbering{gobble}
\setlength{\tabcolsep}{10pt}
\renewcommand{\arraystretch}{1.7}
\usepackage{indentfirst}
 
\begin{document}

\section*{\hfil B. Fibonacci\hfil}

\begin{center}
\begin{tabular}{ |cc| } 
 \hline
 Time Limit & 1 detik \\ 
 \hline
 Memory Limit & 256 MB \\
 \hline
\end{tabular}
\end{center}

\subsection*{Deskripsi}

\par\noindent Barisan fibonacci merupakan suatu barisan bilangan bulat dimana setiap suku fibonacci merupakan penjumlahan dua suku sebelumnya. Secara matematis, bilangan fibonacci didefinisikan sebagai berikut:
\begin{align*}
F_n & = F_{n-1} + F_{n-2} \\
F_1 & = F_2 = 1
\end{align*}
\par\noindent Diberikan sebuah bilangan bulat N, tentukan nilai $F_N$ mod $10^9+7$.

\subsection*{Format Masukan}

\par\noindent Baris pertama berisi sebuah bilangan bulat $T$ yaitu banyaknya kasus.
\par\noindent $T$ baris berikutnya berisi sebuah bilangan bulat $N$.

\subsection*{Format Keluaran}

\par\noindent Setiap kasus keluarkan sebuah bilangan yaitu nilai $F_N$.

\subsection*{Contoh Masukan}

\begin{lstlisting}
2
1
5
\end{lstlisting}

\subsection*{Contoh Keluaran}

\begin{lstlisting}
1
5
\end{lstlisting}


\subsection*{Batasan}

\begin{itemize}
	\item $1 \leq T \leq 1000$
	\item $1 \leq N \leq 1000$
\end{itemize}

\end{document}
