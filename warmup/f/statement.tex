
\documentclass{article}
\usepackage{amsmath}
\usepackage{listings}
\usepackage[margin=1in]{geometry}
\pagenumbering{gobble}
\setlength{\tabcolsep}{10pt}
\renewcommand{\arraystretch}{1.7}
\usepackage{indentfirst}
 
\begin{document}

\section*{\hfil F. Terdekat di Antara yang Terjauh\hfil}

\begin{center}
\begin{tabular}{ |cc| } 
 \hline
 Time Limit & 2 detik \\ 
 \hline
 Memory Limit & 256 MB \\
 \hline
\end{tabular}
\end{center}

\subsection*{Deskripsi}
\par\noindent
Diberikan sebuah tree dengan $N$ node. Setiap node dinamai dari $1$ hingga $N$. Setiap node pada tree memiliki salah satu warna dari $1$ hingga $K$ (inklusif). Mungkin saja ada lebih dari $1$ node yang memiliki warna yang sama, atau bisa saja ada warna yang tidak terdapat pada tree tersebut. 
\newline\par\noindent
Terdapat dua fungsi $F$ dan $G$. Kedua fungsi tersebut didefinisikan sebagai berikut:
\begin{enumerate}
\item $F(C, X)$ menyatakan jarak terjauh dari sebuah node $X$ ke node yang memiliki warna $C$. Apabila tidak ada node yang memiliki warna $C$, $F(C, X)$ bernilai $0$.
\item $G(C)$ menyatakan minimum dari $F(C, X)$ untuk setiap node $1 \leq X \leq N$. Tentukan $G(C)$ untuk setiap $1 \leq C \leq K$!
\end{enumerate}

\subsection*{Format Masukan}
\par\noindent
Baris pertama berisi dua buah bilangan bulat $N$ dan $K$.
\newline\par\noindent
Baris kedua berisi $N$ buah bilangan bulat $C_i$ ($1 \leq i \leq N$).
\newline\par\noindent
$N - 1$ baris berikutnya masing-masing berisi 2 buah bilangan bulat $x_i$ dan $y_i$ yang menandakan bahwa terdapat jalan dua arah yang menghubungkan node $x_i$ dan $y_i$.

\subsection*{Format Keluaran}
\par\noindent Sebuah baris berisi $K$ buah bilangan bulat $G(1)$, $G(2)$, ..., $G(K)$.

\subsection*{Contoh Masukan}

\begin{lstlisting}
3 2
1 2 1
1 2
1 3
\end{lstlisting}

\subsection*{Contoh Keluaran}

\begin{lstlisting}
1 0
\end{lstlisting}


\subsection*{Batasan}

\begin{itemize}
	\item $1 \leq N \leq 100.000$
    \item $1 \leq K \leq N$
    \item $1 \leq C_i \leq K$
    \item $1 \leq x_i, y_i \leq N$
    \item $x_i \neq y_i$
    \item Dijamin graf yang diberikan merupakan sebuah pohon
    
\end{itemize}

\end{document}
