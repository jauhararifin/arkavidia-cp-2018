
\documentclass{article}
\usepackage{amsmath}
\usepackage{listings}
\usepackage[margin=1in]{geometry}
\pagenumbering{gobble}
\setlength{\tabcolsep}{10pt}
\renewcommand{\arraystretch}{1.7}
\usepackage{indentfirst}
 
\begin{document}

\section*{\hfil E. Path Ganjil\hfil}

\begin{center}
\begin{tabular}{ |cc| } 
 \hline
 Time Limit & 1 detik \\ 
 \hline
 Memory Limit & 256 MB \\
 \hline
\end{tabular}
\end{center}

\subsection*{Deskripsi}
\par\noindent
Lelah membaca deskripsi soal  \textit{warmup} Arkavidia yang terlalu panjang, Anda memutuskan untuk mencari soal dengan deskripsi pendek. Beruntung Anda menemukan soal ini!
\newline
\par\noindent
Diberikan sebuah \textit{undirected weighted graph}.  \textit{Path} didefinisikan sebagai jejak telusur  \textit{edge} dimulai dari suatu  \textit{vertex} dan berakhir pada  \textit{vertex} lain tanpa ada  \textit{edge} yang dilewati lebih dari sekali. Ongkos suatu  \textit{path} didefinisikan sebagai total  \textit{weight edge} yang dilewati pada  \textit{path} tersebut. 
\newline
\par\noindent
Berapa jumlah ongkos dari seluruh  \textit{path} yang mungkin yang memiliki ongkos bernilai ganjil?

\subsection*{Format Masukan}
\par\noindent Baris pertama berisi sebuah bilangan bulat $N$, banyaknya  \textit{vertex} pada  \textit{tree}. $N-1$ baris berikutnya masing-masing berisi $3$ buah bilangan bulat $U_i$, $V_i$, $C_i$, artinya ada  \textit{undirected edge} yang menghubungkan  \textit{vertex} ke $U_i$ dan $V_i$ dengan  \textit{weight} $C_i$.

\subsection*{Format Keluaran}

\par\noindent Keluarkan sebuah bilangan bulat, jumlah ongkos dari seluruh  \textit{path} yang mungkin yang memiliki ongkos bernilai ganjil.

\subsection*{Contoh Masukan 1}
\begin{lstlisting}
3
1 2 2
2 3 1
\end{lstlisting}
\subsection*{Contoh Keluaran 1}
\begin{lstlisting}
8
\end{lstlisting}

\subsection*{Contoh Masukan 2}
\begin{lstlisting}
4
1 2 1
2 3 4
2 4 9
\end{lstlisting}
\subsection*{Contoh Keluaran 2}
\begin{lstlisting}
56
\end{lstlisting}

\subsection*{Penjelasan}
\par\noindent
Pada contoh kasus pertama, ada 9 kemungkinan  \textit{path}:
\par\noindent
1 $\rightarrow$ 1, biaya 0
\par\noindent
1 $\rightarrow$  2, biaya 2
\par\noindent
1 $\rightarrow$  2 $\rightarrow$  3, biaya 3
\par\noindent
2 $\rightarrow$  1, biaya 2
\par\noindent
2 $\rightarrow$  2, biaya 0
\par\noindent
2 $\rightarrow$  3, biaya 1
\par\noindent
3 $\rightarrow$  2 $\rightarrow$  1, biaya 3
\par\noindent
3 $\rightarrow$  2, biaya 1
\par\noindent
3 $\rightarrow$  3, biaya 0
\newline
\par\noindent
Total  \textit{path} dengan ongkos ganjil adalah: $3 + 1 + 3 + 1 = 8$

\subsection*{Batasan}

\begin{itemize}
	\item $1 \leq N \leq 10^5$
    \item $1 \leq U_i, V_i, \leq N$
    \item $1 \leq C_i \leq 100$
\end{itemize}

\end{document}
