
\documentclass{article}
\usepackage{amsmath}
\usepackage{listings}
\usepackage{graphicx}
\usepackage[margin=1in]{geometry}
\pagenumbering{gobble}
\setlength{\tabcolsep}{10pt}
\renewcommand{\arraystretch}{1.7}
 
\begin{document}

\section*{\hfil G. Random Generator\hfil}

\begin{center}
\begin{tabular}{ |cc| } 
 \hline
 Time Limit & 2 detik \\
 \hline 
 Memory Limit & 256 MB \\
 \hline
\end{tabular}
\end{center}

\subsection*{Deskripsi}

\par\noindent Bocan iseng membuat formula untuk random generator buatannya, yakni menggunakan rekurens:

\begin{center}
$a_n = p \times a_{n-1} + q$
\end{center}

\par\noindent Kini, ia memiliki bilangan $x$. Bocan penasaran, berapa $n$ terkecil supaya $a_n \mod m = x$?

\subsection*{Format Masukan}

\par\noindent Baris pertama berisi bilangan bulat $p$, $q$, dan $a_0$ dipisahkan dengan spasi.
\par\noindent Baris kedua berisi bilangan bulat $m$ dan $x$.

\subsection*{Format Keluaran}

\par\noindent Keluarkan satu baris berisi bilangan bulat $n$ terkecil yang mungkin. Jika tidak ada, keluarkan $-1$.

\subsection*{Contoh Masukan 1}

\begin{lstlisting}
2 1 5
7 1
\end{lstlisting}

\subsection*{Contoh Keluaran 1}

\begin{lstlisting}
4
\end{lstlisting}

\subsection*{Contoh Masukan 2}

\begin{lstlisting}
2 0 2
4 1
\end{lstlisting}

\subsection*{Contoh Keluaran 2}

\begin{lstlisting}
-1
\end{lstlisting}

\subsection*{Contoh Masukan 3}

\begin{lstlisting}
7 9 5
11 5
\end{lstlisting}

\subsection*{Contoh Keluaran 3}

\begin{lstlisting}
0
\end{lstlisting}

\subsection*{Batasan}

\begin{itemize}
  \item $2 \leq m \leq 10^{12}$
  \item $0 \leq p, q, a_0, x < m$
  \item $m$ dijamin merupakan bilangan prima
\end{itemize}

\subsection*{Penjelasan}

\par\noindent Pada contoh pertama, barisan yang terbentuk adalah $5, 11, 23, 47, 99,$ dst. Pada $n = 4$, $99 \mod 7 = 1$. Tidak ada nilai $n$ lebih kecil yang memenuhi.

\par\noindent Pada contoh kedua, tidak ada $n$ yang memenuhi.

\par\noindent Pada contoh ketiga, nilai $a_0$ sudah memenuhi.

\end{document}