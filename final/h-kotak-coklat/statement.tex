
\documentclass{article}
\usepackage{amsmath}
\usepackage{listings}
\usepackage{graphicx}
\usepackage[margin=1in]{geometry}
\usepackage{hyperref}
\pagenumbering{gobble}
\setlength{\tabcolsep}{10pt}
\renewcommand{\arraystretch}{1.7}
 
\begin{document}

\section*{\hfil H. Kotak Coklat\hfil}

\begin{center}
\begin{tabular}{ |cc| } 
 \hline
 Time Limit & 2 detik \\
 \hline 
 Memory Limit & 256 MB \\
 \hline
\end{tabular}
\end{center}

\subsection*{Deskripsi}

\includegraphics[width=10cm]{box-of-chocolate}

\par\noindent Bocan suka coklat. Turpa, sebagai fans berat Bocan, memberi hadiah sekotak coklat berukuran $N \times N$. Tiap coklat berada pada koordinat $(x,y)$ yang berbeda. Supaya tidak cepat habis, Bocan ingin bermain-main terlebih dahulu dengan coklatnya.

\par\noindent Bocan melakukan $Q$ aksi. Tiap aksi, ia dapat:

\begin{itemize}
	\item mengambil coklat di koordinat $(x,y)$.
	\item menaruh coklat di koordinat $(x,y)$.
	\item menghitung coklat di segiempat yang kedua ujungnya koordinat $(x_1,y_1)$ dan $(x_2,y_2)$
\end{itemize}

\par\noindent Setiap koordinat di kotak coklat hanya dapat menampung $1$ buah coklat.

\subsection*{Format Masukan}

\par\noindent Baris pertama berisi sebuah bilangan bulat $N$ dan $Q$.
\par\noindent $Q$ baris berikutnya berisi salah satu dari:
\begin{itemize}
	\item \lstinline{1 x y}, menaruh coklat di $(x,y)$
	\item \lstinline{2 x y}, mengambil coklat di $(x,y)$
	\item \lstinline{3 x1 y1 x2 y2}, menghitung coklat di segiempat $(x_1,y_1)$ hingga $(x_2,y_2)$
\end{itemize}
\par\noindent Untuk setiap aksi ambil \lstinline{x y}, dijamin ada coklat di koordinat $(x,y)$. Begitu pula untuk setiap aksi taruh \lstinline{x y}, dijamin tdak ada coklat di koordinat $(x,y)$.

\subsection*{Format Keluaran}

\par\noindent Untuk tiap query hitung, keluarkan sebuah baris berisi jumlah coklat di segiempat yang kedua ujungnya $(x_1,y_1)$ dan $(x_2,y_2)$.

\subsection*{Contoh Masukan}

\begin{lstlisting}
10 7
1 1 1
1 2 1
1 3 3
1 6 5
3 1 1 5 4
2 2 1
3 1 1 4 5
\end{lstlisting}

\subsection*{Contoh Keluaran}

\begin{lstlisting}
3
2
\end{lstlisting}

\subsection*{Batasan}

\begin{itemize}
  \item $1 \leq N \leq 10^9$
  \item $1 \leq Q \leq 10^5$
  \item $1 \leq x, y, x_1, y_1, x_2, y_2 \leq N$
  \item $x_1 \leq x_2$ dan $y_1 \leq y_2$
\end{itemize}

\subsection*{Penjelasan}

\par\noindent Sebelum aksi hitung pertama, sudah ada $4$ coklat, di $(1,1)$, $(2,1)$, dan $(3,3)$, dan $(6,5)$. Tiga coklat pertama berada di dalam segiempat yang ujungnya $(1,1)$ dan $(4,5)$, sehingga dikeluarkan $3$.

\par\noindent Sebelum aksi hitung kedua, coklat pada koordinat $(2,1)$ diambil sehingga tersisa $2$ coklat pada hasil perhitungan.

\par\noindent Sumber gambar:\url{http://www.lazybite.com/82-large_default/signature-wooden-chocolate-box.jpg}

\end{document}
