
\documentclass{article}
\usepackage{amsmath}
\usepackage{listings}
\usepackage{graphicx}
\usepackage[margin=1in]{geometry}
% \pagenumbering{gobble}
\usepackage{fancyhdr}
\setlength{\tabcolsep}{10pt}
\renewcommand{\arraystretch}{1.7}
 
\begin{document}

\section*{\hfil A. Mengurutkan Bilangan\hfil}

% \begin{center}
% \begin{tabular}{ |cc| } 
%  \hline
%  Time Limit & 2 detik \\
%  \hline 
%  Memory Limit & 256 MB \\
%  \hline
% \end{tabular}
% \end{center}

\pagestyle{fancy}
\fancyhf{}
\renewcommand{\headrulewidth}{0pt}
\fancyfoot[CE,CO]{A - \thepage}

\subsection*{Deskripsi}

\par\noindent Diberikan suatu array $A$ yang berisi $N$ elemen yang dinomori dari $1$ sampai $N$. Elemen ke-$i$ pada $A$ dinyatakan dengan $A_i$. Setiap $A_i$ terdiri dari $2$ bilangan bulat $X_i$ dan $Y_i$. Terdapat fungsi $F(a,b)$ yang mengembalikan hasil perkalian dari $a$ dan $b$. Sebagai contoh, $F(2,3)$ mengembalikan nilai $6$, sedangkan $F(7,3)$ mengembalikan nilai $21$. Terdapat juga array $B$ yang berisi elemen-elemen array $A$ yang telah diurutkan. $A_i$ memiliki posisi lebih awal dari $A_j$ pada $B$ apabila salah satu hal berikut terpenuhi:

\begin{itemize}
	\item $F(X_i,Y_i) > F(X_j, Y_j)$.
	\item $F(X_i,Y_i) = F(X_j, Y_j)$ dan $i < j$.
\end{itemize}

\par\noindent Terdapat juga fungsi $G(i)$ yang mengembalikan posisi $A_i$ pada $B$. Hitung nilai dari $G(1), G(2), ... , G(N)$. 

\subsection*{Format Masukan}

\par\noindent Baris pertama berisi sebuah bilangan bulat $N$. $N$ baris berikutnya masing-masing berisi dua buah bilangan bulat yang menyatakan nilai $X_i$ dan $Y_i$.

\subsection*{Format Keluaran}

\par\noindent Keluarkan $N$ baris. Baris ke-$i$ menyatakan nilai $G(i)$.

\subsection*{Contoh Masukan}

\begin{lstlisting}
5
1 5
2 4
3 3
4 2
5 1
\end{lstlisting}

\subsection*{Contoh Keluaran}

\begin{lstlisting}
4
2
1
3
5
\end{lstlisting}

\subsection*{Penjelasan}

\par\noindent Array $B$ pada contoh masukan diatas:
\begin{lstlisting}
3 3
2 4
4 2
1 5
5 1
\end{lstlisting}

\subsection*{Batasan}

\begin{itemize}
	\item $1 \leq N \leq 5.000$
	\item $0 \leq X_i, Y_i \leq 10^{5.000}$
	\item Terdapat suatu bilangan $C$ sehingga untuk setiap $i$, $X_i + Y_i = C$
\end{itemize}

\end{document}
