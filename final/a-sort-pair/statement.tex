
\documentclass{article}
\usepackage{amsmath}
\usepackage{listings}
\usepackage{graphicx}
\usepackage[margin=1in]{geometry}
\pagenumbering{gobble}
\setlength{\tabcolsep}{10pt}
\renewcommand{\arraystretch}{1.7}
 
\begin{document}

\section*{\hfil A. Mengurutkan Bilangan\hfil}

\begin{center}
\begin{tabular}{ |cc| } 
 \hline
 Time Limit & 2 detik \\
 \hline 
 Memory Limit & 256 MB \\
 \hline
\end{tabular}
\end{center}

\subsection*{Deskripsi}

\par\noindent Sebuah array $A$ memili elemen berupa pasangan bilangan bulat. Pasangan bilangan dinyatakan dalam notasi $(X,Y)$. Nilai dari sebuah pasangan bilangan $(X,Y)$ adalah $X \times Y$. Pasangan bilangan $(2,3)$ memiliki nilai $6$, sedangkan pasangan bilangan $(7,3)$ memiliki nilai $21$. $A_i = (X_i, Y_i)$ menyatakan pasangan bilangan ke-$i$ pada array $A$.

\par\noindent Diberikan array $A$ dengan $N$ buah elemen dimulai dari $1$ hingga $N$. Urutkan array $A$ sehingga nilai dari setiap pasangan bilangan terurut mengecil. Jika ada pasangan bilangan yang nilainya sama, pasangan bilangan yang lebih dulu ditentukan berdasarkan index awal pasangan bilangan pada array $A$.

\subsection*{Format Masukan}

\par\noindent Baris pertama berisi sebuah bilangan bulat $N$ yang menyatakan banyaknya elemen pada array $A$.
\par\noindent $N$ baris berikutnya berisi dua buah bilangan yang menyatakan nilai $X_i$ dan $Y_i$.

\subsection*{Format Keluaran}

\par\noindent Keluarkan $N$ baris dengan baris ke-$i$ menyatakan posisi pasangan bilangan ke-$i$ setelah diurutkan.

\subsection*{Contoh Masukan}

\begin{lstlisting}
4
1 4
2 3
3 2
4 1
\end{lstlisting}

\subsection*{Contoh Keluaran}

\begin{lstlisting}
3
1
2
4
\end{lstlisting}

\subsection*{Batasan}

\begin{itemize}
	\item $1 \leq N \leq 50000$
	\item $1 \leq X_i, Y_i \leq 10^{100}$
	\item Terdapat suatu bilangan $C$ sehingga untuk setiap $i$ berlaku $X_i + Y_i = C$
\end{itemize}

\end{document}
