
\documentclass{article}
\usepackage{amsmath}
\usepackage{listings}
\usepackage{graphicx}
\usepackage[margin=1in]{geometry}
\pagenumbering{gobble}
\setlength{\tabcolsep}{10pt}
\renewcommand{\arraystretch}{1.7}
 
\begin{document}

\section*{\hfil E. Meretas Password Wifi\hfil}

% \begin{center}
% \begin{tabular}{ |cc| } 
%  \hline
%  Time Limit & 2 detik \\
%  \hline 
%  Memory Limit & 256 MB \\
%  \hline
% \end{tabular}
% \end{center}

\subsection*{Deskripsi}

\par\noindent Bocan sedang mencoba meretas password wifi temannya dengan \textit{bruteforce}. Agar dapat melakukan percobaan password secara paralel, Bocan menggunakan $N$ buah laptop yang dinomori dari $1$ sampai $N$. Program di tiap laptop tidak dijalankan pada waktu yang bersamaan. Kecepatan masing-masing laptop berbeda. Tepatnya, laptop ke-$i$ dapat mencoba $d_i$ password tiap detik.

\par\noindent Setelah beberapa waktu, Bocan sadar dia tidak mencatat waktu eksekusi yang sudah dilalui masing-masing laptop. Karena itu, Ia mereset stopwatchnya, kemudian mencatat jumlah password yang sudah dicoba oleh masing-masing laptop. Jumlah password yang telah dicoba pada laptop ke-$i$ dinyatakan sebagai $s_i$. Dengan kata lain, $s_i$ menyatakan jumlah password yang telah dicoba oleh laptop ke-$i$ pada detik ke-$0$. 

\par\noindent Bocan memiliki $Q$ buah pertanyaan. Pada pertanyaan ke-$i$, Ia ingin tahu berapakah jumlah password terbanyak yang telah dicoba oleh masing-masing laptop, untuk laptop bernomor $l_i$ sampai $r_i$ (inklusif) pada detik ke-$t_i$. Diasumsikan program Bocan belum selesai pada $t_i$ untuk tiap $i$.

\subsection*{Format Masukan}

\par\noindent Baris pertama berisi sebuah bilangan bulat $N$.
\par\noindent $N$ baris berikutnya berisi dua buah bilangan bulat dengan baris ke-$i$ menyatakan nilai $s_i$ dan $d_i$.
\par\noindent Pada baris berikutnya terdapat bilangan $Q$ menyatakan jumlah pertanyaan.
\par\noindent $Q$ baris berikutnya berisi pertanyaan, dengan setiap baris berisi tiga buah bilangan bulat $l_i$, $r_i$, dan $t_i$ seperti yang dijelaskan pada deskripsi soal.

\subsection*{Format Keluaran}

\par\noindent Keluarkan $Q$ baris dengan baris ke-$i$ berisi jawaban pertanyaan ke-$i$ menyatakan jumlah password terbanyak yang sudah dicoba laptop bernomor antara $l_i$ sampai $r_i$ (inklusif) pada detik ke-$t_i$.

\subsection*{Contoh Masukan}

\begin{lstlisting}
4
30 2
15 4
10 4
1 5
3
1 4 1
2 4 1
1 4 20
\end{lstlisting}

\subsection*{Contoh Keluaran}

\begin{lstlisting}
32
19
101
\end{lstlisting}

\subsection*{Batasan}

\begin{itemize}
  \item $1 \leq N, Q \leq 10^5$
  \item $1 \leq l_i \leq r_i \leq N$
  \item $1 \leq s_i, d_i, t_i \leq 10^9$
\end{itemize}

\end{document}
