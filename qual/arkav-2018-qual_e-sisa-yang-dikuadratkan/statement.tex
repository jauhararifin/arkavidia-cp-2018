
\documentclass{article}
\usepackage{amsmath}
\usepackage{listings}
\usepackage[margin=1in]{geometry}
\pagenumbering{gobble}
\setlength{\tabcolsep}{10pt}
\renewcommand{\arraystretch}{1.7}
 
\begin{document}

\section*{\hfil E. Sisa yang Dikuadratkan\hfil}

\begin{center}
\begin{tabular}{ |cc| } 
 \hline
 Time Limit & 2 detik \\ 
 \hline
 Memory Limit & 256 MB \\
 \hline
\end{tabular}
\end{center}

\subsection*{Deskripsi}
Diberikan sebuah array $A$ berisi $N$ bilangan bulat. $A_{i}$ menandakan elemen $A$ ke-$i$ untuk $1 \leq i \leq N$. Terdapat $Q$ buah \textit{query}. Setiap query diberikan sebuah bilangan bulat $X$. Hitunglah nilai dari:

\begin{gather*} 
(A_{1} \bmod X)^2 + (A_{2} \bmod X)^2 + \dots + (A_{N} \bmod X)^2
\end{gather*} 

\subsection*{Format Masukan}
\par Baris pertama berisi sebuah bilangan $T$ yang menyatakan banyaknya kasus.
\newline Baris pertama setiap kasus berisi 2 bilangan bulat: $N$ dan $Q$.
\newline Baris kedua setiap kasus berisi $N$ bilangan bulat. Bilangan ke-$i$ menyatakan nilai $A_{i}$.
\newline $Q$ baris berikutnya dari setiap kasus berisi sebuah bilangan bulat $X$ seperti yang dijelaskan pada deskripsi soal.

\subsection*{Format Keluaran}

\par Untuk setiap kasus, keluarkan $Q$ baris bilangan bulat yang menyatakan jawaban tiap \textit{query}.

\subsection*{Contoh Masukan}
\begin{lstlisting}
1
5 3
1 100 7 33 20
1
5
123456789
\end{lstlisting}

\subsection*{Contoh Keluaran}
\begin{lstlisting}
0
14
11539
\end{lstlisting}

\subsection*{Penjelasan}

\par Terdapat satu kasus uji. Dalam kasus uji tersebut, N = 5 dan Q = 3. Pada query kedua dari kasus uji tersebut, hasilnya adalah:
\begin{gather*} 
(1 \bmod 5)^2 + (100 \bmod 5)^2 + (7 \bmod 5)^2 + (33 \bmod 5)^2 + (20 \bmod 5)^2 = 1^2 + 0^2 + 2^2 + 3^2 + 0^2 = 14
\end{gather*} 

\subsection*{Batasan}

\begin{itemize}
	\item $1 \leq T \leq 10$
	\item $1 \leq N \leq 2 \times 10^5$
	\item $1 \leq Q \leq 2 \times 10^5$
	\item $1 \leq A_i \leq 2 \times 10^5$
	\item $1 \leq X \leq 10^9$
\end{itemize}

\end{document}
