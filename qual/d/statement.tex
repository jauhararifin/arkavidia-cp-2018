
\documentclass{article}
\usepackage{amsmath}
\usepackage{listings}
\usepackage[margin=1in]{geometry}
\pagenumbering{gobble}
\setlength{\tabcolsep}{10pt}
\renewcommand{\arraystretch}{1.7}
 
\begin{document}

\section*{\hfil D. Fahar, Jundi, dan Kotak\hfil}

\begin{center}
\begin{tabular}{ |cc| } 
 \hline
 Time Limit & 2 detik \\ 
 \hline
 Memory Limit & 256 MB \\
 \hline
\end{tabular}
\end{center}

\subsection*{Deskripsi}

\par Fahar dan Jundi adalah teman yang sangat akrab. Mereka juga suka bermain dengan angka. Mereka mempunyai teka-teki untuk Anda. Beginilah teka-tekinya... 
\newline
\newline Fahar mempunyai $N$ bilangan bulat. Bilangan ke-$i$ bernilai $A_{i}$. Jundi mempunyai kotak yang banyaknya tak berhingga. Mereka ingin memasukkan bilangan-bilangan ke dalam kotak-kotak tersebut dengan konfigurasi sebagai berikut:

\begin{enumerate}
  \item Sebuah bilangan hanya dapat dimasukkan tepat ke satu kotak.
  \item Setiap kotak terdiri atas minimal $K$ angka.
\end{enumerate}

\par\noindent Terdapat fungsi $F$ yang menghitung selisih bilangan terbesar dan terkecil dalam suatu kotak. Terdapat pula fungsi $G$ yang menghitung total dari $F$ untuk semua kotak yang terisi oleh bilangan. Fahar dan Jundi penasaran, dari semua konfigurasi yang mungkin, berapakah nilai $G$ minimum yang mungkin? Karena kemungkinannya sangat banyak, mereka kewalahan. Bantulah mereka mencari nilai $G$ minimum yang mungkin!

\subsection*{Format Masukan}
\par Baris pertama berisi sebuah bilangan bulat $T$, yaitu banyaknya kasus uji.
\newline Setiap kasus uji terdiri dari 2 baris.
\newline Baris pertama setiap kasus uji berisi 2 bilangan bulat $N$ dan $K$.
\newline Baris kedua setiap kasus uji berisi $N$ bilangan bulat $A_{i}$.

\subsection*{Format Keluaran}

\par Untuk setiap kasus uji, keluarkan sebuah baris yang berisi nilai $G$ minimum yang mungkin untuk kasus uji yang bersangkutan.

\subsection*{Contoh Masukan}

\begin{lstlisting}
3
5 3
1 1 2 1 2
5 2
1 1 2 1 2
10 3
1 2098 2145 35 25 23 2112 23 2123 13
\end{lstlisting}

\subsection*{Contoh Keluaran}

\begin{lstlisting}
1
0
81
\end{lstlisting}

\subsection*{Penjelasan}

Pada kasus uji pertama, tidak mungkin untuk membagi bilangan ke lebih dari 1 kotak, sehingga nilai $G = 2 - 1 = 1$.
\newline 
\newline Pada kasus uji kedua, salah satu solusi optimal yang mungkin adalah:
\begin{itemize}
\item Kotak 1 : $[1, 1, 1]$, $F = 1 - 1 = 0$
\item Kotak 2 : $[2, 2]$, $F = 2 - 2 = 0$
\end{itemize}

\par\noindent Maka nilai $G$ untuk konfigurasi di atas adalah $0 + 0 = 0$.
\newline
\newline Berikut adalah konfigurasi yang kurang optimal untuk kasus uji kedua:
\begin{itemize}
\item Kotak 1 : $[1, 1]$, $F = 1 - 1 = 0$
\item Kotak 2 : $[2, 1, 2]$, $F = 2 - 1 = 1$
\end{itemize}

\par\noindent Nilai $G$ untuk konfigurasi diatas adalah $0 + 1 = 1$, sehingga bukan merupakan solusi optimal.

\subsection*{Batasan}

\begin{itemize}
	\item $1 \leq T \leq 20$
	\item $1 \leq K \leq N\leq 100.000$
	\item $1 \leq A_i\leq10^9$
\end{itemize}

\end{document}
