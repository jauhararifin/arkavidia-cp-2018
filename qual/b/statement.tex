
\documentclass{article}
\usepackage{amsmath}
\usepackage{listings}
\pagenumbering{gobble}
 
\begin{document}

\section*{\hfil Prefix XOR\hfil}

\subsection*{Deskripsi}
\par Setelah memelajari prefix sum, Turfa ingin mencoba untuk melakukan prefix xor. Ia kemudian membuah sebuah array $A$ berisi $N$ buah bilangan bulat. Setelah melakukan precomputing nilai xor, akhirnya Turfa merasa senang karena dapat menjawab query yang berisi dua buah bilangan bulat yaitu $l$ dan $r$ dan menghasilkan nilai xor dari seluruh elemen $A$ dari $l$ hingga $r$, atau dengan kata lain Turfa dapat menghitung nilai $A_l \oplus A_{l+1} \dots A_r$. Dimana $a \oplus b$ berarti $a$ xor $b$.

\par Kesenangan Turfa tidak berlangsung lama setelah Agung memberinya query yang lebih sulit. Diberikan dua buah bilangan yaitu $l$ dan $r$, tentukan nilai dari $l \oplus (l+1) \oplus (l+2) \oplus \dots \oplus r$.

\subsection*{Format Masukan}
\par Baris pertama berisi sebuah bilangan $T$ yang menyatakan banyaknya kasus
\par Setiap kasus terdiri dari sebuah baris berisi dua buah bilangan bulat $l$ dan $r$.

\subsection*{Format Keluaran}

\par Untuk setiap kasus, keluarkan sebuah bilangan bulat yang menyatakan nilai dari $l \oplus (l+1) \oplus (l+2) \oplus \dots \oplus r$.

\subsection*{Contoh Masukan}

\begin{lstlisting}
3
1 2
7 10
100 101

\end{lstlisting}

\subsection*{Contoh Keluaran}

\begin{lstlisting}

3
12
1

\end{lstlisting}

\subsection*{Batasan}

\begin{itemize}
	\item $1 \leq T \leq 200000$
	\item $1 \leq l < r \leq 10^{18}$
\end{itemize}

\end{document}
