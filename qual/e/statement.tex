
\documentclass{article}
\usepackage{amsmath}
\usepackage{listings}
\usepackage[margin=1in]{geometry}
\pagenumbering{gobble}
\usepackage{indentfirst}
 
\begin{document}

\section*{\hfil Sisa yang Dikuadratkan\hfil}

\begin{center}
\begin{tabular}{ |cc| } 
 \hline
 Time Limit & 2 detik \\ 
 Memory Limit & 256 MB \\
 \hline
\end{tabular}
\end{center}

\subsection*{Deskripsi}

\par Diberikan sebuah array $A$ berisi $N$ bilangan bulat. $A_{i}$ adalah elemen $A$ ke-$i$ dengan $1 \leq i \leq N$.

\par Terdapat $Q$ buah \textit{query}. Setiap query diberikan sebuah bilangan bulat $X$. Hitunglah nilai dari:
\begin{gather*} 
(A_{1} \bmod X)^2 + (A_{2} \bmod X)^2 + ... + (A_{n} \bmod X)^2
\end{gather*} 

\subsection*{Format Masukan}
\par Baris pertama berisi sebuah bilangan T yang menyatakan banyaknya kasus.
\par Baris pertama setiap kasus berisi dua bilangan bulat: $N$ sebagai panjang array $A$, dan $Q$ banyak \textit{query}.
\par Baris kedua berisi $N$ buah bilangan bulat yang menyatakan nilai $A$.
\par $Q$ baris berikutnya berisi sebuah bilangan bulat $X$ seperti yang dijelaskan pada deskripsi soal.

\subsection*{Format Keluaran}

\par Untuk setiap kasus, keluarkan $Q$ baris bilangan bulat yang menyatakan jawaban tiap query.

\subsection*{Contoh Masukan}

\begin{lstlisting}
1
5 3
1 100 7 33 20
1
5
123456789
\end{lstlisting}

\subsection*{Contoh Keluaran}

\begin{lstlisting}
0
14
11539
\end{lstlisting}

\subsection*{Penjelasan}

\par Pada query kedua:
\begin{gather*} 
(1 \bmod 5)^2 + (100 \bmod 5)^2 + (7 \bmod 5)^2 + (33 \bmod 5)^2 + (20 \bmod 5)^2 = 1^2 + 0^2 + 2^2 + 3^2 + 0^2 = 14
\end{gather*} 

\subsection*{Batasan}

\par Terdapat paling banyak 10 kasus ($1 \leq T\leq 10$). Untuk setiap kasus, berlaku batasan berikut:
\begin{itemize}
	\item $1 \leq N\leq 2\times 10^5$
	\item $1 \leq Q\leq 2\times 10^5$
	\item $1 \leq A_i\leq 2\times 10^5$
	\item $1 \leq X \leq 10^9$
\end{itemize}

\end{document}
