
\documentclass{article}
\usepackage{amsmath}
\usepackage{listings}
\usepackage[margin=1in]{geometry}
\pagenumbering{gobble}
 
\begin{document}

\section*{\hfil Kue Ulang Tahun\hfil}

\begin{center}
\begin{tabular}{ |cc| } 
 \hline
 Time Limit & 1 detik \\ 
 Memory Limit & 256 MB \\
 \hline
\end{tabular}
\end{center}

\subsection*{Deskripsi}
\par Pak Ansah memiliki kue yang akan dibagikan menjadi $N$ potong. Setiap potongan dapat memiliki besaran yang berbeda-beda. Potongan ke-$i$ memiliki besaran $A_i$ unit. Pak Ansah juga mempunyai N anak (kebetulan sekali!) sehingga ia ingin membagikan kue ke anak-anaknya dengan aturan sebagai berikut:

\begin{enumerate}
\item Anak ke-$i$ mendapatkan potongan kue ke-$i$.
\item Setiap anak tidak boleh makan kue dengan besaran lebih banyak dari yang saudara-saudaranya.
\item Setiap anak boleh makan sebagian dari potongan kue yang dimiliki. Dengan kata lain, boleh saja tidak memakan potongan kue secara utuh. 
\item Setiap anak tidak boleh membagikan kue ke saudaranya. 
\end{enumerate}
Tentukan banyaknya total besaran minimum yang tersisa setelah anak-anaknya memakan kue sesuai dengan aturan diatas.

\subsection*{Format Masukan}
Baris pertama berisi sebuah bilangan bulat T, yaitu banyaknya kasus uji. \newline
Untuk setiap kasus uji, berisi sebuah bilangan bulat N, diikuti dengan N bilangan bulat. Bilangan ke-$i$ menyatakan $A_i$, yaitu besarnya potongan kue ke-$i$. 

\subsection*{Format Keluaran}

\par Untuk setiap kasus uji, keluarkan satu baris yang berisi jawaban untuk kasus uji yang bersangkutan.

\subsection*{Contoh Masukan}

\begin{lstlisting}
3
5 1 2 3 4 5
3 7 2 5
10 7 7 7 7 7 7 7 7 7 7
\end{lstlisting}

\subsection*{Contoh Keluaran}

\begin{lstlisting}
10
8
0
\end{lstlisting}

\subsection*{Penjelasan}

Untuk kasus ketiga, total besaran kue = $70$ unit. Seluruh anak mendapatkan potongan dengan besaran yang sama, sehingga setiap anak mendapatkan $7$ unit, dan total kue yang tersisa = $70 - 7 * 10 = 0$ unit tersisa.

\subsection*{Batasan}

\begin{itemize}
	\item $1 \leq T\leq 100$
	\item $1 \leq N\leq 10^5$
	\item $1 \leq A_i\leq 10^6$
\end{itemize}

\end{document}
