
\documentclass{article}
\usepackage{amsmath}
\usepackage{listings}
\usepackage[margin=1in]{geometry}
\usepackage{hyperref}
\pagenumbering{gobble}
\setlength{\tabcolsep}{10pt}
\renewcommand{\arraystretch}{1.7}
 
\begin{document}

\section*{\hfil B. XOR\hfil}

\begin{center}
\begin{tabular}{ |cc| } 
 \hline
 Time Limit & 2 detik \\ 
 \hline
 Memory Limit & 256 MB \\
 \hline
\end{tabular}
\end{center}

\subsection*{Deskripsi}
\par Turfa baru belajar tentang operasi bitwise \href{https://en.wikipedia.org/wiki/Exclusive_or}{XOR} (Exclusive-OR). Operasi bitwise XOR ditandai dengan simbol $\oplus$. Turfa juga mempunyai 2 buah bilangan $l$ dan $r$. Turfa ingin mengetahui nilai dari $l \oplus (l + 1) \oplus (l + 2) \dots \oplus (r - 1) \oplus r$. Bantulah Turfa!

\subsection*{Format Masukan}
\par Baris pertama berisi sebuah bilangan bulat $T$, yaitu banyaknya kasus uji.
\newline Setiap kasus terdiri dari sebuah baris berisi dua buah bilangan bulat $l$ dan $r$.

\subsection*{Format Keluaran}

\par Untuk setiap kasus uji, keluarkan sebuah baris yang berisi jawaban sesuai deskripsi diatas. 

\subsection*{Contoh Masukan}

\begin{lstlisting}
4
1 1
1 2
7 10
100 101

\end{lstlisting}

\subsection*{Contoh Keluaran}

\begin{lstlisting}
1
3
12
1

\end{lstlisting}

\subsection*{Batasan}

\begin{itemize}
	\item $1 \leq T\leq 200.000$
	\item $1 \leq l \leq r \leq 10^{18}$
\end{itemize}

\end{document}
