
\documentclass{article}
\usepackage{amsmath}
\usepackage{listings}
\usepackage[margin=1in]{geometry}
\usepackage{hyperref}
\pagenumbering{gobble}
\setlength{\tabcolsep}{10pt}
\renewcommand{\arraystretch}{1.7}
 
\begin{document}

\section*{\hfil C. AND\hfil}

\begin{center}
\begin{tabular}{ |cc| } 
 \hline
 Time Limit & 2 detik \\ 
 \hline
 Memory Limit & 256 MB \\
 \hline
\end{tabular}
\end{center}

\subsection*{Deskripsi}
\par Setelah belajar XOR, Turfa kemudian belajar tentang operasi \textit{bitwise} AND (\url{https://en.wikipedia.org/wiki/Bitwise_operation#AND}). Operasi \textit{bitwise} AND ditandai dengan simbol $\land$. Turfa mempunyai sebuah array $A$ berisi $N$ buah bilangan bulat. $A_{i}$ menyatakan elemen $A$ ke-$i$ dengan $1 \leq i \leq N$.
\par\noindent Selain itu, Turfa mempunyai $Q$ \textit{query}. Setiap \textit{query} diberikan dua buah bilangan bulat $L$ dan $R$. Turfa ingin mengetahui nilai dari $A_{L} \land A_{L+1} \land \dots \land A_{R-1} \land A_{R}$. Bantulah Turfa!

\subsection*{Format Masukan}
\par Baris pertama berisi sebuah bilangan $T$ yang menyatakan banyaknya kasus. Setiap kasus uji memiliki format sebagai berikut:
\begin{itemize}
\item Baris pertama berisi sebuah bilangan bulat: $N$ sebagai panjang array $A$.
\item Baris kedua berisi $N$ buah bilangan bulat. Bilangan ke-$i$ menyatakan nilai $A_{i}$.
\item Baris ketiga berisi sebuah bilangan bulat $Q$ yang menyatakan banyak \textit{query}.
\item $Q$ baris berikutnya berisi dua buah bilangan bulat $L$ dan $R$ seperti yang dijelaskan pada deskripsi soal.
\end{itemize}

\subsection*{Format Keluaran}

\par Untuk setiap kasus uji, keluarkan $Q$ baris yang berisi jawaban sesuai deskripsi di atas. 

\subsection*{Contoh Masukan}

\begin{lstlisting}
1
7
7 4 5 6 31 58 1
4
1 5
3 4
1 7
4 6
\end{lstlisting}

\subsection*{Contoh Keluaran}

\begin{lstlisting}
4
4
0
2

\end{lstlisting}

\subsection*{Batasan}

\begin{itemize}
  \item $1 \leq T\leq 10$
  \item $1 \leq N,\space Q\leq 200.000$
  \item $0 \leq A_{i} \leq 10^9$
  \item $1 \leq L \leq R \leq N$
  \item Jumlah $N$ dalam satu file tidak lebih dari $200.000$. Jumlah $Q$ dalam satu file tidak lebih dari $200.000$.
\end{itemize}

\end{document}
